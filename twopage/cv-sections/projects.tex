% ----------------------------------------------------------------------------------------
% SECTION TITLE
% ----------------------------------------------------------------------------------------

\cvsection{Projects}

% ----------------------------------------------------------------------------------------
% SECTION CONTENT
% ----------------------------------------------------------------------------------------

\begin{cventries}

  % ------------------------------------------------

  
  \cventry
  {Open Source Project by \textbf{AOSSIE}}
  {\href{https://gitlab.com/aossie/CarbonFootprint-API}{\entrytitlestyle{CarbonFootprint-API}}{}}
  {}
  {Nov. 2018, Present}
  {
    \begin{cvitems}
      \item This project aims to build a RESTful API, that is the one place to go to for any information that you require on Carbon Emissions.
      \item Doing regular contributions to enhance the code quality and adding new features to the project.
      \item My contributions include adding a new feature which helps user track his daily CO2 emission and compare with the average emission value of the day.
      \item Also helping to restart the CI pipeline which was failing due to problems in the API.
      \item \textbf{Tech Stack:} MERN stack, Mocha and Supertest for testing, Redis for caching.
    \end{cvitems}
  }

  \cventry
  {Ethereum blockchain project}
  {\href{https://github.com/thakursaurabh1998/certification-validation}{\entrytitlestyle{Certify Validate}}
    {}}
  {}
  {Dec. 2018 - Present}
  {
    \begin{cvitems}
      \item Developed an open source application which can be used by any organization to generate and validate certificates.
      \item Certificate data is stored in the blockchain and can be used to verify and validate the authenticity of the certificate.
      % \item Currently deployed on Rinkeby Test Network for the prototype.
      \item Continous integration set up with Docker and Heroku.
      \item \textbf{In proceedings} to be included in a reputed open source organization.
      \item \textbf{Tech Stack:} MERN stack and Truffle framework.
    \end{cvitems}
  }

  \cventry
  {Decentralized Voting App}
  {\href{https://github.com/thakursaurabh1998/voting-dapp}{\entrytitlestyle{Let's Vote}}}
  {}
  {Oct. 2018}
  {
    \begin{cvitems}
      \item A perfect solution to reduce the problems faced during elections by various authorities.
      \item A voter can't vote twice and his identity is always secured, also provides data according to regions.
      \item A robust and secure way to count votes along side biometric authentication with help of aadhar number and face recognition.
      \item \textbf{Tech Stack:} Ethereum Blockchain | Express(REST API) | React.
    \end{cvitems}
  }


  \cventry
  {Decentralized app for making donations}
  {\href{https://github.com/thakursaurabh1998/NGO-Helper}{\entrytitlestyle{NGO-Helper}}}
  {}
  {Sep. 2018}
  {
    \begin{cvitems}
      \item Created an app where donors and people in need can get together directly.
      \item All the donations are managed with a blockchain in a decentralized manner.
      \item Each transaction made is saved in the blockchain hence working as a public ledger.
      \item \textbf{Tech Stack:} Express(REST API) | Blockchain created in NodeJS | React | heroku | IBMcloud
    \end{cvitems}
    \vspace{5mm}
  }

  \cventry
  {Notes sharing webapp}
  {\href{https://github.com/thakursaurabh1998/endprep-django}{\entrytitlestyle{Endprep}}}
  {}
  {Mar. 2018 - Jun. 2018}
  {
    \begin{cvitems}
      \item The main motive of this project is notes sharing.
      \item Students can upload and share notes.
      \item All the data gets uploaded and stored to the server.
      \item Students get a link through which they can directly share the documents with their peers.
      \item \textbf{Tech Stack:} Django | jQuery | Bootstrap | FacebookOAuth | GoogleComputeEngine
    \end{cvitems}
    \vspace{5mm}
  }

  % ------------------------------------------------
\end{cventries}

%%% Local Variables:
%%% mode: xelatex
%%% TeX-master: "../resume_twopage.tex"
%%% End:
